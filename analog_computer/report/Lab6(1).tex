\documentclass[11pt]{article}
\usepackage[letterpaper]{geometry}
\usepackage{times}
\usepackage{verbatim}
\usepackage{graphicx}
\usepackage{float}
\usepackage{fullwidth}
\title{ENGR-241 Lab }
\author{}
\geometry{top=1.0in, bottom=1.0in, left=1.0in, right=1.0in}
\begin{document}
\subsection*{Overview}
	For this lab we became familiar with an AMF analog computer. We produced a few output signals and explored how the analog computer was built. We then designed a new analog computer with less components that produced the same effects. The instructions include finding: f=a(b-10c) using the sum/difference op amps, and to output the inverting integration of the input for the integrating op amp. We then describe the purpose of the SET and 1.0 SEC---CONT functions. Lastly, we design new circuits of just one summing/difference and one integrating amplifier to produce the same results, describe the designs, and to state the advantages and disadvantages of digital vs analog models. 
\subsubsection*{Basic Topology}
We had used the schematic from the analog computer as a reference for making our own design. We used KiCad to build and simulate the new design, shown below. 
	\subsection*{Data Tables}
All of the calculations and readings for the circuits are tabulated here.\\
\subsubsection*{Circuit}
\begin{table}[h]
	\def\arraystretch{1.2}%
	\begin{tabular}{|l|l|l|l|l|l|l|}
		\hline
		Summing(a(b-10c))	& a(input) 		& b(input)	& c(input)	& Output (calculated)		& Output (Observed) 	& \% Diff			\\ \hline
		Test 1		& 1.0			& 0.454V	& 0.417V	& -3.72V					& -3.73V				&0.0722\%				\\ \hline
		Test 2		& 1.0			& 0.775V	& 0.671V	& -5.94V					& -5.96V				&0.3367\%				\\ \hline
		Test 3		& 1.0			& 1.004V	& 0.848V	& -7.48V					& -7.53V				&0.6684\%				\\ \hline
	
		
	\end{tabular}
\end{table}

\begin{table}[h]
	\def\arraystretch{1.2}%
	\begin{tabular}{|l|l|l|l|l|l|l|}
		\hline
		Integration(inverting)	& Input 		& Output(calculated)	& Output (Observed)	 		& \% Diff			\\ \hline
		Test 1					& 0.497V		& -0.497V				& -0.4V						&19.52\%			\\ \hline
		Test 2					& 1.009V		& -1.009V				& -0.87V					&13.78\%			\\ \hline
		Test 3					& 2.01V			& -2.001V				& -1.8V						&10.045\%			\\ \hline
		
		
	\end{tabular}
\end{table}

\section*{Design}
	The analog computer was state-of-the-art in its day, but compared to today's digital models is somewhat clunky, simple and inefficient. The new design attempts to produce the same results with fewer components, making the 	design more efficient.
\subsubsection*{Operational Amplifier}
	At the core of the circuit is the op-amp. We chose a Hitachi HA17741 op-amp\footnote{\href{http://product.ic114.com/PDF/H/e204043_ha17741.pdf}{HA17741 Datasheet}} simply because it's what was readily available in the lab. This is a clone of the ubiquitous LM741 op-amp that is pin compatible with it. The important limitations of this opamp are the supply voltage limits of $\pm 18v$, the output voltage limits of $\pm 14v$ with output load resistance $>10k \Omega$ and $\pm 10v$ with output load of $2k \Omega$.

\subsection*{Procedure}
	We demonstrated the ability of the AMF Computer to perform several math operations as described in the assignment requirements. THe first test utilized the summing and difference Op Amp and the second showed the function of the integrating Op Amp. For each test we measured the output with a fluke connected to the output jacks.
\subsection*{Summing ($f=a(b-10c)$)}
	For this portion of the lab we supplied power to a 1x addition jack and a 10x subtraction jack and we set the potentiometer to a scaling factor of 1("a" value). The "b" value was set with the addition input and the "c" value was set with the subtracting input. We set the inputs at three different values and calculated expected output for the chosen input values. The data is recorded in the data tables above.
\subsection*{Integrating ($f=\int$$f(t) {d}t$)}
	To perform the integration we connected a constant input to the integrating circuit and performed the tests over one second using the "1.0 SEC" mode described below to set our limits of integration. We performed three tests and compared the results to our calculated values, which are recorded in the data table above. 
		
\subsection*{Competitor's Product Analysis}
	
	The Educational Computer runs off of a standard 60Hz power source and converts this for the circuits op amps VCC values of +/- 15V. It has a variable input for each of its integrating and summing boards. This model has three summing and difference Op Amps and two integrating Op Amps. \\
	
	Each of the integrating Op Amps take in a input voltage and performs and inverting integration of the input. There are three separate input plugs, two are will input the chosen value, and the third will amplify the input by a factor of ten. The initial conditions are set with the red button(labeled SET) which holds the voltage constant while adjusting the initial condition knob. For the tests we performed on this model sent the output values to a fluke, verifying the circuit functioned properly. There are two different modes of integration on the AMF Educational Computer, the computer has a switch that can select either continuous integration (labeled CONT) or an integration over 1 second(labeled 1.0 SEC.). We performed tests in both modes and had moderately accurate results considering the age of the equipment. For the tests we performed on this model, we connected the output to a fluke, verifying the circuit functioned properly. The data can be found in the "Integration" data tables above. \\
	
	All three of the summing and difference amplifiers take an input voltage and add or subtract there values depending on where the inputs are connected to the circuit. There are a total of 6 input plugs, three adding and three subtracting. Like the integrating input, there are two  that directly input in the chosen input voltage and one that scales the input by ten. This is true for both the adding and subtracting inputs. The circuit also has a potentiometer that scales the output between zero and one and it is adjusted by a knob on the computer. We performed multiple tests and had highly accurate results. For the tests we performed on this model, we connected the output to a fluke, verifying the circuit functioned properly. The data can be found in the "Summing" data tables above.

\subsection*{Circuit Analysis}
	The circuit we have designed is similar in funtionality to the Analog Computer described above. Our design incorporates a single summing and difference Op Amp and a single integrating Op Amp. Each of the circuits have 741 Op Amps that are supplied with a +/- 15V VCC and the both have three input and output jacks.  The input jacks are connected to ground when not in use to prevent current from flowing through them while not in use. There are two "1x" inputs and one "10x" input for each circuit, similar to the AMF computer.
	
	The integrating circuit, shown above uses a feedback capacitor to perform the integration by charging and discharging as the input of the circuit changes. Resistor and Capacitance values were chosen to provide an output that is one times the integral and it is inverting because the feedback is negative. The resistor placed in parallel is installed to slow the time at which the Op Amp will reach saturation. Large resistor values were chosen for the circuit to minimize the effect of noise on the system in our design.
	
	The addition and subtraction Op Amp takes the input voltages and uses the inverting and non-inverting terminals of the Op Amp to perform addition and subtraction. Once again, large resistor values were used to minimize the influence of parasitic capacitance and inductance on the circuit. 
\subsection*{Design Advantages}	
	The analog computer we have designed has several advantages over the AMF Computer. The main advantage being that we cut costs by minimizing the amount of materials we need to produce our computer by only including the two Op Amps. Additionally, we produced easy to read schematics with properly labeled pins on the Op Amp unlike our competitor's product.
 \section*{Summary}
\subsection*{SET and 1.0 SEC---CONT Functions}
	The SET button, when pushed it showed a constant value. This constant value would 'reset' and show the output at the set initial condition. Then when released, the output returned to its normal function over time. 
	The 1.0 SEC---CONT toggle had 3 slots: 1.0 SEC (left), off (middle), and CONT (right). The 1.0 SEC slot would perform the function desired for one second, then it would stay constant at the final value at t(1.0).
	The CONT slot allows the output to perform the function desired continuously until the op amps saturate.  
	
\subsection *{Analog vs. Digital}
	For the analog computer design, some pros include: outputs can be any real values, or some irrational values like pi; the resulting voltage is instantaneously found and it can calculate high values or repetitive calculations quickly. Some cons with the analog design are: that it serves a single purpose, and it is more susceptible to noise. 
	For the digital computer design, which is more common today, the pros include: that it can be reprogrammed to perform other tasks; and fine tuning allows for less noise. The cons include: that it can only give you quantified, fixed values or approximations of values; for repeated calculations, it is harder to be processed, but due to greater advancements in processors, this is now as fast as or faster than an analog design. Overall, digital computers are more heavily used as a result of their versatility, and our boss who insists that digital is 'just a fad' will be sorely mistaken.  \\
	\newpage
\section*{Appendix A - Simplified Design Adjustments}
	When we were first designing our new analog computer, going off the schematic was somewhat of a challenge due to the fact that it was hand drawn. We used KiCad to design and model the schematic, and the integrating op amp was the most work to get it to operate the same as the original. 
	Several adjustments such as increasing the resistor to cancel noise, input and output voltage symbols, switches and potentiometers were added or omitted, and even editing the KiCad code itself was needed in order to produce the proper output. Once this was said and done, KiCad was able to simulate the schematic and show the same output voltages that were calculated above within the table. 
	A picture below shows the simulated output from the integrating op amp, being fed a square wave signal in red with the output triangle wave in green.
	 
	
\end{document}