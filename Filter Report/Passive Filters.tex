\documentclass[11pt]{article}
\usepackage[letterpaper]{geometry}
\usepackage{times}
\usepackage{verbatim}
\usepackage{graphicx}
\usepackage{float}
\usepackage{fullwidth}
\usepackage{amsmath}
\usepackage{amssymb}
\usepackage{fourier}
\usepackage{hyperref}
\graphicspath{{Images/}}
\title{ENGR-241 Passive Filters Lab}
\author{Jeremy Munson, Lauren Speirs \& Andrew Henrikson}
\geometry{top=.8in, bottom=.8in, left=.8in, right=.8in}

\setlength{\parindent}{0em}
\setlength{\parskip}{.5em}
\begin{document}
	\maketitle
	\subsection*{Overview}
	For this lab we designed and  constructed high and low pass filter circuits with RC components. We designed each circuit with a cutoff frequency of 10 KHz and used the signal generator to produce a 2V p-p sinusoidal waveform as the voltage source. We then swept the frequency between 1KHz and 100KHz and calculated the expected output voltage for given frequencies.
	\subsection*{Circuit Diagrams}
	  
	\subsection*{Calculations}
	The calculations for this circuit 